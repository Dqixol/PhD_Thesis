%%%%%%%%%%%%%%%%%%%%%%%%%%%%%%%%%%%%%%%%%%%
%%%              ttH                    %%%
%%%%%%%%%%%%%%%%%%%%%%%%%%%%%%%%%%%%%%%%%%%
\section[\ttH production]{\ttH}
\label{subsec:ttH}

\subsection[Powheg+Pythia8]{\POWPY[8]}
%\label{subsubsec:ttH_PP8}

Nominal \(t\bar{t}H\) samples are produced with \POWPY[8].
The \hdamp value is set to \(\qty{352.5}{\GeV} = 3/4 \cdot (\mH+2\mtop)\).

\paragraph{Samples}
%\label{par:ttH_PP8_samples}
\Cref{tab:ttH_PP8} gives the nominal \ttH samples.

\begin{table}[htbp]
  \caption{Nominal \ttH samples produced with \POWPY[8].}%
  \label{tab:ttH_PP8}
  \centering
  \begin{tabular}{l l}
    \toprule
    DSID range & Description \\
    \midrule
    346343 & \ttH, \(H\to\) all, \(\ttbar\to\) all-hadronic \\
    346344 & \ttH, \(H\to\) all, \(\ttbar\to\) semileptonic \\
    346345 & \ttH, \(H\to\) all, \(\ttbar\to\) dileptonic \\
    346525 & \ttH, \(H\to \gamma\gamma\), \(\ttbar\to\) all \\
    \bottomrule
  \end{tabular}
\end{table}

\paragraph{Short description:}

The production of \ttH events was modelled using the
\POWHEGBOX[v2]~\cite{Frixione:2007nw,Nason:2004rx,Frixione:2007vw,Alioli:2010xd,Hartanto:2015uka}
generator at NLO with the \NNPDF[3.0nlo]~\cite{Ball:2014uwa} PDF set.
The events were interfaced to \PYTHIA[8.230]~\cite{Sjostrand:2014zea}~using
the A14 tune~\cite{ATL-PHYS-PUB-2014-021} and the
\NNPDF[2.3lo]~\cite{Ball:2014uwa} PDF set. The decays of bottom and charm hadrons
were performed by \EVTGEN[1.6.0]~\cite{Lange:2001uf}.

\paragraph{Long description:}

The production of \ttH events was modelled using the
\POWHEGBOX[v2]~\cite{Frixione:2007nw,Nason:2004rx,Frixione:2007vw,Alioli:2010xd,Hartanto:2015uka}
generator, which provided matrix elements at next-to-leading order (NLO) in the strong coupling
constant \alphas in the five-flavour scheme with the \NNPDF[3.0nlo]~\cite{Ball:2014uwa} PDF set.
The functional form of the renormalisation and factorisation scales was
set to \(\sqrt[3]{m_\text{T}(t)\cdot m_\text{T}(\bar{t}) \cdot m_\text{T}(H)}\).
The events were interfaced to \PYTHIA[8.230]~\cite{Sjostrand:2014zea}
using the A14 tune~\cite{ATL-PHYS-PUB-2014-021} and the
\NNPDF[2.3lo]~\cite{Ball:2014uwa} PDF set. The decays of bottom and charm hadrons
were performed by \EVTGEN[1.6.0]~\cite{Lange:2001uf}.

The cross-section was calculated at NLO QCD and NLO EW accuracy using
\MGNLO as reported in Ref.~\cite{deFlorian:2016spz}.
The predicted value at \(\rts = \qty{13}{\TeV}\) is
\(507^{+35}_{-50}\,\unit{\fb}\), where the uncertainties were estimated from
variations of \alphas and the renormalisation and factorisation scales.

The uncertainty in the initial-state radiation (ISR) was estimated using the Var3c
up/down variations of the A14 tune. Uncertainties due to missing
higher-order corrections were evaluated through simultaneous variations of the
renormalisation and factorisation scales by factors of
2.0 and 0.5. Uncertainties in the PDFs were evaluated using the 100
variations of the \NNPDF[3.0nlo] set.


\subsection[Powheg+Herwig7]{\POWHER[7]}
%\label{subsubsec:ttH_PH7}

\paragraph{Samples}
%\label{par:ttH_PH7_samples}

\Cref{tab:ttH_PH7} presents alternative \ttH samples.

\begin{table}[htbp]
  \caption{Alternative \ttH \POWHER[7] samples produced to evaluate systematic uncertainties
  due to different MC models for parton showering and hadronisation.}%
  \label{tab:ttH_PH7}
  \centering
  \begin{tabular}{l l}
    \toprule
    DSID range & Description \\
    \midrule
    346346 & \ttH, \(H\to\) all, \(\ttbar\to\) all-hadronic \\
    346347 & \ttH, \(H\to\) all, \(\ttbar\to\) semileptonic \\
    346348 & \ttH, \(H\to\) all, \(\ttbar\to\) dileptonic \\
    346526 & \ttH, \(H\to\gamma\gamma\), \(\ttbar\to\) all \\
    \bottomrule
  \end{tabular}
\end{table}

\paragraph{Short description:}

The impact of using a different parton shower and hadronisation model was evaluated by showering the nominal hard-scatter events with
\HERWIG[7.04]~\cite{Bahr:2008pv,Bellm:2015jjp} using the H7UE set of tuned parameters~\cite{Bellm:2015jjp} and
the \MMHT[lo] PDF set~\cite{Harland-Lang:2014zoa}.

\paragraph{Long description:}

The impact of using a different parton shower and hadronisation model was evaluated
by comparing the nominal sample with another sample produced with
the \POWHEGBOX[v2]~\cite{Frixione:2007nw,Nason:2004rx,Frixione:2007vw,Alioli:2010xd}
generator but interfaced with \HERWIG[7.04]~\cite{Bahr:2008pv,Bellm:2015jjp}, using the
H7UE set of tuned parameters~\cite{Bellm:2015jjp} and
the \MMHT[lo] PDF set~\cite{Harland-Lang:2014zoa}.
\POWHEGBOX provided matrix elements at next-to-leading
order~(NLO) in the strong coupling constant \alphas with the
\NNPDF[3.0nlo]~\cite{Ball:2014uwa} parton distribution function~(PDF).
The functional form of the renormalisation and factorisation scales was
set to \(\sqrt[3]{m_\text{T}(t)\cdot m_\text{T}(\bar{t}) \cdot m_\text{T}(H)}\).
The decays of bottom and charm hadrons
were simulated using the \EVTGEN[1.6.0] program~\cite{Lange:2001uf}.
