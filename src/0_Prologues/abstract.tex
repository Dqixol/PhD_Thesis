\begin{abstract} \begin{spacing}{1.25}
    In this thesis, I present significant advancements in the reconstruction and identification of highly boosted pairs of tau leptons within the ATLAS experiment, 
    alongside a search for the Graviton in the \(HH\rightarrow bb\tau\tau\) decay channel. The development of a muon-removal method for boosted 
    \(\tau_\mu\tau_{\text{had}}\) reconstruction has improved the identification efficiency of hadronically decaying taus in the 
    presence of nearby muons. This method recovers \(\tau_{\text{had}}\) identification efficiency to levels expected for isolated decays 
    across all working points, as well as restoring the precision of kinematic measurements for the visible \(\tau_{\text{had}}\) system. 
    Benchmarking with \(Z \rightarrow \tau_\mu\tau_{\text{had}}\) samples from the complete Run-2 dataset recorded by the ATLAS detector 
    affirmed the robustness of this method, showing agreement between data and Monte Carlo simulations. 
    Similarly, the electron-removal method for boosted \(\tau_e\tau_{\text{had}}\) reconstruction markedly improved the accurate 
    reconstruction of visible decay products of the tau lepton pairs within a single jet by removing the nearby electron contamination.
    Utilising the muon-removal advancements, we conducted a search for the Graviton in the \(HH\rightarrow bb\tau\tau\) channel using the full 
    ATLAS Run-2 dataset collected at \(\sqrt{s} = 13 \text{TeV}\) with an integrated luminosity of 140 fb\(^{-1}\). 
    Enhanced by a GNN-based \(bb\)-jet tagging algorithm and the muon-removal technique, our event selection process 
    achieved high signal efficiency. Results showed good agreement between data and Monte Carlo simulations in the control 
    region, demonstrating negligible QCD background contamination in the signal region. Evaluation of statistical and 
    systematic uncertainties led to the derivation of 95\% confidence level limits on the production cross-section
    \(\sigma(pp \rightarrow G \rightarrow HH)\) for mass points ranging from 2 to 5 TeV, significantly surpassing 
    previous ATLAS searches in the \(HH\rightarrow bb\tau\tau\) channel. 
    These advancements in tau lepton reconstruction and the subsequent Graviton search highlight the potential to 
    open up a new phase-space, enhancing the sensitivity of the ATLAS 
    experiment for discovering new physics phenomena in the highly boosted di-\(\tau\) channel.
\end{spacing}\end{abstract}
