%!TEX root = ../thesis.tex
%*******************************************************************************
%****************************** Third Chapter **********************************
%*******************************************************************************
\chapter{Conclusion}

\graphicspath{{18Chap8_Concl/Figs/}}

In this thesis, we have explored multiple facets of tau lepton reconstruction and identification within the ATLAS detector 
framework, as well as conducted a detailed search for the Graviton in the \(bb\tau\tau\) decay channel. 

The muon-removal method developed for the boosted \(\tau_\mu\tau_{\text{had}}\) reconstruction has demonstrated significant 
improvements in the identification efficiency of hadronically decaying taus in the presence of a nearby muon. This method 
recovers the \(\tau_{\text{had}}\) identification efficiency to levels expected for isolated \(\tau_{\text{had}}\) decays 
across all TauID working points, ensuring that the measurement precision for the kinematic properties of the visible 
\(\tau_{\text{had}}\) system is maintained. This advancement was validated through benchmarking with a highly boosted 
\(Z \rightarrow \tau_\mu\tau_{\text{had}}\) sample from the complete Run-2 dataset recorded by the ATLAS detector. The 
results showed good agreement between data and Monte Carlo simulations, highlighting the robustness of the method.

The electron-removal method for boosted \(\tau_e\tau_{\text{had}}\) reconstruction similarly displayed significant improvements. 
This method accurately reconstructs the visible decay products within a single \(\tau_{\text{seed}}\) jet, addressing challenges 
posed by the presence of nearby electrons. This advancement enhances the signal yield, and in the mean time suppresses the background contamination. 
The method's effectiveness was benchmarked using the \(Z \rightarrow \tau_e\tau_{\text{had}}\) 
channel.

In our search for the Graviton in the \(bb\tau\tau\) channel, we utilised the full ATLAS Run-2 dataset, collected at 
\(\sqrt{s} = 13 \text{ TeV}\) with an integrated luminosity of 140 fb\(^{-1}\). This search benefited from the 
advancements in ATLAS combined performance, particularly the GNN-based \(bb\)-jet tagging algorithm and the muon-removal 
technique for \(\tau_{\text{had}}\) identification. Our results showed good agreement between data 
and Monte Carlo simulations in the control region, indicating negligible QCD background contamination in the signal 
region. Statistical and systematic uncertainties were evaluated and incorporated into the upper limits.

The expected 95\% confidence level (CL) upper limits on the \(\sigma(pp \rightarrow G \rightarrow HH)\) were derived for mass 
points ranging from 2 to 5 TeV. The limits obtained from this analysis significantly outperformed 
previous ATLAS searches in the \(bb\tau\tau\) channel. For instance, at a resonance mass of 3000 GeV, we achieved 
an limit of 3.8~fb, compared to the previous limit of approximately 55~fb in the 
\(HH \rightarrow bb\tau_{\text{had}}\tau_{\text{had}}\) channel. Despite the 20-times lower $\hhbbtmth$ branching ratio,
our limits were only 5-times less stringent than those obtained in the $\hhbbbb$ channel. This highlights the
performance of the muon-removal technique in the $\tau_{\text{had}}$ identification, and the success of this analysis in
suppressing the SM background contamination.

In conclusion, the advancements in tau lepton reconstruction presented in this thesis 
open up a new final state in the search for new physics phenomena. 
The results of the Graviton search in the \(bb\tau\tau\) channel
demonstrate the potential of these advancements to significantly enhance the discovery potential of the ATLAS experiment
in the highly boosted di-$\tau$ channel.