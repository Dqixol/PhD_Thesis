%!TEX root = ../thesis.tex

\chapter{Introduction}

The field of particle physics is a vibrant and rapidly evolving discipline that seeks to 
uncover the fundamental constituents of matter and the forces that govern their interactions. 
At the heart of this endeavour lies the Standard Model (SM) of particle physics~\cite{ParticleDataGroup:2022pth}, a robust 
theoretical framework that has successfully explained a wide array of experimental results. 
The SM encompasses the electromagnetic, weak, and strong forces, which are mediated by gauge 
bosons, and the Higgs mechanism, which imparts mass to these particles. However, despite 
its successes, the SM is known to be incomplete, prompting the exploration of extensions 
that can address its limitations.

The Standard Model provides a comprehensive description of the elementary particles and 
their interactions~\cite{RevModPhys.71.S96}. It classifies all known fundamental particles into two groups: 
fermions, which make up matter, and bosons, which mediate the forces between fermions. 
The fermions include quarks and leptons, while the bosons include the photon, W and Z bosons, 
and gluons. The Higgs boson, discovered in 2012 at the Large Hadron Collider (LHC)~\cite{Evans:2008zzb}, 
is a pivotal component of the SM, responsible for giving mass to the other fundamental particles 
through the Higgs mechanism.

Symmetry principles, particularly gauge invariance, underpin the SM. These principles dictate 
the interactions between particles, leading to the formulation of electroweak theory 
for electromagnetic interactions and weak interactions, quantum chromodynamics (QCD) for strong interactions.
Both of these theories has been rigorously tested
and confirmed through numerous experiments.

Despite its successes, the SM does not include gravity~\cite{Carlip_2015}, which is described by General Relativity 
in the classical regime. The quest for a quantum theory of gravity, which would unify all fundamental 
forces, remains one of the most significant challenges in theoretical physics. Various approaches, 
such as string theory and loop quantum gravity, aim to address this gap, but a complete and 
experimentally verified theory is yet to be established. Additionally, the SM does not account 
for dark matter and dark energy, which constitute approximately 95\%
of the universe's energy density. The existence of these components is inferred from astrophysical 
observations, such as the rotation curves of galaxies and the accelerated expansion of the universe. 
Extensions to the SM, including supersymmetry and theories involving extra dimensions, have been proposed 
to incorporate these phenomena, but experimental confirmation is still pending.

This thesis is dedicated to exploring the intricacies of the Standard Model and its potential gravitational extensions, as will be introduced in Chapter~2.
In the latter part of Chapter~2, the modelling of the proton-proton collisions at the Large Hadron Collider (LHC) will be discussed.
Chapter~3 will introduce the ATLAS detector, a sophisticated instrument designed to measure the 
properties of particles produced in high-energy collisions. The thesis will provide a comprehensive 
overview of the ATLAS detector's components, including the inner detector, calorimeters, muon 
spectrometer, and the trigger and data acquisition system. This chapter will also discuss the
reconstruction of particles within the ATLAS detector. 
In Chapter~4, special attention will be given to the reconstruction and identification of tau leptons, which 
play a crucial role in various searches for new physics. Innovative methods for tau lepton reconstruction, 
including machine learning techniques such as Recurrent Neural Networks (RNNs), will be discussed. 
The thesis will then present the development and performance evaluation of muon- and electron-removal 
methods in boosted $\tlth$ lepton-pair reconstruction in Chapter~5 and Chapter~6
respectively, along with benchmarks using $\Zttlephad$ events.
To put these new developments to test, in Chapter~7, the thesis will explore searches for heavy resonant particles. In particular, a hypothetical Graviton, 
in channels involving tau leptons and b-jets. These searches are motivated by theories that extend the SM and aim to provide insights into the nature of gravity.
Finally, Chapter~8 will summarise this thesis's key findings and discuss their implications for the field of particle physics.
This thesis aims to contribute to the ongoing efforts to understand the fundamental 
structure of the universe by leveraging the capabilities of the ATLAS experiment and exploring 
both established and novel aspects of particle physics.

While the theoretical underpinnings presented in Chapter~2 are well-established in the community, 
and the design and construction of the ATLAS detector described in Chapter 3 were carried out 
by the broader ATLAS Collaboration, I did not contribute directly to the formulation of these 
theories nor the detector hardware. 
My principal contributions begin in Chapter~5 and extend through Chapter~7, where I served as 
the primary analyst. Specifically, I led the development and performance evaluation of 
muon- and electron-removal methods for boosted $\tau$-lepton pair reconstruction, 
benchmarks using $\Zttlephad$ events, and the subsequent searches for heavy resonant 
particles such as hypothetical gravitons in final states involving $\tau$-leptons and 
$b$-jets. These analyses were performed in close collaboration with other members of 
the ATLAS team.
Although I was responsible for designing and executing the primary analyses described 
in those chapters, the work benefited significantly from the ATLAS software frameworks, 
simulation tools, and ongoing discussions within the collaboration.